\documentclass[12pt,letterpaper]{article}
\usepackage[latin1]{inputenc}
\usepackage[spanish]{babel}
\usepackage{amsmath}
\usepackage{amsfonts}
\usepackage{amssymb}
\usepackage{makeidx}
\usepackage{graphicx}
\usepackage[left=2cm,right=2cm,top=2cm,bottom=2cm]{geometry}
\author{Nicol�s Gonz�lez Mart�nez}
\title{Logaritmos}
\begin{document}
\maketitle
\textbf{Definici�n:} El logaritmo de un n�mero real positivo $b$ en base $a$, es el n�mero $m$ a que se debe elevar la base para obtener dicho n�mero, es decir:
\begin{center}
$log_{a}(b)=m \Leftrightarrow a^{m}=b$ donde $b>1$ y $a>0 , a \not = 1$
\end{center}
Observaciones:
\begin{enumerate}
\item La expresi�n $log_{a}(b)=m$ se lee "logaritmo en base $a$ de $b$ es $m$
\item el logaritmo es la operaci�n inversa de la exponencial
\item $log_{10}(a)=log(a)$
\end{enumerate}
\textbf{Propiedades de Logaritmos}
\begin{enumerate}
\item $lob_{a}(b \cdot c) = lob_{a}(b)+lob_{a}(c)$
\item $log_{a} \left( \dfrac{b}{c} \right) = log_{a}(b) - log_{a}(c)$
\item $log_{a}(b^{n})= n \cdot log_{a}(b)$
\item $log_{a}(\sqrt[n]{b})= \dfrac{1}{n}log_{a}(b)$ de donde se concluye que $log_{a}(\sqrt[n]{b^{m}})= \dfrac{m}{n}log_{a}(b)$ 
\item $log_{a}(b)= \dfrac{log_{c}(b)}{log_{c}(a)}$
\item $log_{a}(1)=0$
\item $log_{e}(b)=ln(b)$ siendo $e$ el n�mero de Euler.
\end{enumerate}
\textbf{Calcule el valor de:}
\begin{enumerate}
\item $log_{3}(242)$
\item $log_{2}(512)$
\item $log_{5}(625)$
\item $log_{81}(9)$
\item $log_{25}(5)$
\item $log(10000000)$
\item $log_{e}(e^{21})$
\item Exprese en forma de logaritmo la igualdad $4^{3}=63$
\item Exprese en forma de logaritmo la igualdad $12^{2}=144$
\item $5 \cdot log(2 \cdot 2^{-1})$
\item $log_{2} \left( \dfrac{1}{4} \right)$
\item $log_{4}(8)+log_{4}(5)$
\item $log(25)+log(4)$
\item Si $log_{3}(a)-log_{3}(b)=2$, entonces el valor de $\dfrac{a}{b}$ es igual a
\item La expresi�n $log(5)-log(2)+log(6)$ escrita como un solo logaritmo es igual a
\item $log_{2}(32)-log_{2}(64)$
\item $log_{3}(81)-log_{3}(243)+log_{3}(9)$
\item $log_{3} \left( \dfrac{1}{27} \right)$
\item Demuestre que $log_{64}(4)+log_{3}(81)$ es igual a $\dfrac{13}{3}$
\end{enumerate}
\end{document}