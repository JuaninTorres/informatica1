\documentclass[12pt,letterpaper]{article}
\usepackage[latin1]{inputenc}
\usepackage[spanish]{babel}
\usepackage{amsmath}
\usepackage{amsfonts}
\usepackage{amssymb}
\usepackage{makeidx}
\usepackage{graphicx}
\usepackage[left=2cm,right=2cm,top=2cm,bottom=2cm]{geometry}
\author{Nicol�s Gonz�lez Mart�nez}
\title{Definici�n de Conjuntos }
\begin{document}
\maketitle
\begin{enumerate}
\item $\mathbb{N}$: los n�meros naturales poseen dos definiciones \\
Sin cero = $\mathbb{N}^{*}: \lbrace 1,2,3,........ \rbrace \Longleftrightarrow ]0,\infty[$\\
Con cero = $\mathbb{N}: \lbrace 0,1,2,3,........ \rbrace \Longleftrightarrow [0,\infty[$
\item $\mathbb{Z}$: los n�meros enteros se definen como: \\
$\mathbb{Z}= \mathbb{N} \cup -\mathbb{N}$;  donde $-\mathbb{N} = \lbrace -n / n \in \mathbb{N} \rbrace$ \\\\
El conjunto $\mathbb{Z}$ con las propiedades $+$ y $\cdot$ posee la clasificaci�n de grupo, es decir: \\\\
$i)$ es asociativo con la suma: $\Longleftrightarrow (a+b)+c=a+(b+c)$\\
$i')$ es asociativo con la suma: $\Longleftrightarrow (a\cdot b)\cdot c=a\cdot(b\cdot c)$\\\\
$ii)$ existe un elemento neutro al cual llamaremos $e$: $\Longleftrightarrow a+e=a=e+a$\\
$ii')$ existe un elemento neutro al cual llamaremos $e$: $\Longleftrightarrow a\cdot e = a = e\cdot a $\\\\
$iii)$ existe un elemento inverso al cual llamaremos $-a$: $\Longleftrightarrow a+(-a)=e=(-a)+a$\\\\
\textbf{observaci�n:} $\mathbb{Z}$ posee la cualidad de ser conmutativo con la suma, es decir, $\left( \forall x,y \in \mathbb{Z} \right) \left( x+y=y+x \right)$ \\
\textbf{observaci�n:} $\mathbb{Z}$ posee la cualidad de ser conmutativo con el producto, es decir, $\left( \forall x,y \in \mathbb{Z} \right) \left( x\cdot y=y\cdot x \right)$
\item $\mathbb{Q}$: son los n�meros racionales y se definen como: \\
$\mathbb{Q}=  \lbrace \dfrac{p}{q} / p,q \in \mathbb{Z}$ y $q \not = 0 \rbrace$ \\\\
\textbf{observaci�n:} se define la relaci�n $\dfrac{p}{q} \thicksim \dfrac{m}{n} \Longleftrightarrow p \cdot n = q \cdot m$ \\\\
� Como se suma en $\mathbb{Q}$? \\\\
Dados $\dfrac{p}{q} + \dfrac{m}{n}$ esto se resuelve de la siguiente manera: $\dfrac{pn+qm}{qn}$\\\\
� Como se multiplica en $\mathbb{Q}$? \\\\
Dados $\dfrac{p}{q} \cdot \dfrac{m}{n}$ esto se resuelve de la siguiente manera: $\dfrac{p \cdot m}{q \cdot n}$
\item � Quien es $\mathbb{R}$? \\
para la respuesta anterior definiremos los siguiente axiomas: \\\\
\textbf{Axioma 1:} $0,1 \in \mathbb{R}$ donde $0 \not = 1$\\
\textbf{Axioma 2:} $\left( \mathbb{R},+ \right)$ es grupo abeliano \quad (tarea: averiguar que significa Grupo Abeliano)\\
\textbf{Axioma 3:} $\left( \mathbb{R}, \cdot \right)$ es grupo abeliano\\
\textbf{Axioma 4:} Se define la propiedad distributiva: $\left( \forall a,b,c \in \mathbb{R} \right)  \left( a \cdot (b+c)=a\cdot b + a\cdot c \right)$ \\\\
\textbf{Propiedades de $\mathbb{R}$} \\\\
\textbf{1:} $-(a+b)=(-a)+(-b)=-a-b$ \\
\textbf{2:} $-(-a)=a$ \\
\textbf{3:} $a\cdot b$ lo denotaremos desde ahora como $ab$ \\
\textbf{4:} $(ab)^{-1}=a^{-1}b^{-1}$ \\
\textbf{5:} $a0=0$ \\
\textbf{6:} $-(ab)=(-a)b=a(-b)$ \\
\textbf{7:} $(-a)(-b)=ab$ \\
\textbf{8:} $(a^{-1})^{-1}=a$ \\
\textbf{9:} $ab=0 \Longleftrightarrow a=0 \vee b=0$
\end{enumerate}

\end{document}