\documentclass[12pt,letterpaper]{article}
\usepackage[latin1]{inputenc}
\usepackage[spanish]{babel}
\usepackage{amsmath}
\usepackage{amsfonts}
\usepackage{amssymb}
\usepackage{makeidx}
\usepackage{graphicx}
\usepackage[left=2cm,right=2cm,top=2cm,bottom=2cm]{geometry}
\author{Nicol�s Gonz�lez Mart�nez}
\title{Potencias en R}
\begin{document}
\maketitle
\textbf{Potencias en $\mathbb{R}$} \\
Una potencia se define como $ a^{n}=\underbrace{a\cdot a \cdots \cdot a \cdot a}_{{n veces} }$ \\
\textbf{Propiedades de las Potencias}
\begin{enumerate}
\item $a^{n}\cdot a^{m}=a^{n+m}$
\item $a^{n}\div a^{m}=a^{n-m}$
\item $a^{n} \cdot b^{n} = (a\cdot b)^{n}$
\item $a^{n} \div b^{n} = (a \div b)^{n}$ 
\item $(a^{n})^{m}=a^{n\cdot m} = a^{nm}$
\item $a^{-1}=\dfrac{1}{a}$ y esto implica que $a^{-n}=\dfrac{1}{a^{n}}$
\item $a^{0}=1$ y a su vez $a^{1}=a$
\item $\left( \dfrac{1}{a}\right)^{-n}=a^{n}$ y esto implica que $\left( \dfrac{a}{b} \right)^{-n} = \left( \dfrac{b}{a} \right)^{n}$
\item $ \left( \dfrac{a}{b} \right)^{n} = \dfrac{a^{n}}{b^{n}}$
\item $0^{0}$ no esta definido
\end{enumerate}
\textbf{Calcule el valor de:}
\begin{enumerate}
\item $a^{4} \cdot a^{-2} \cdot a^{8}$
\item $4^{2} \cdot 2^{8} \cdot 16^{2}$ 
\item $7^{2} \cdot 7^{9} \div 7^{3}$ 
\item $21^{13} \cdot 21^{7} \div 21^{-2}$
\item $2^{4}\cdot 6^{4} \cdot 4^{4}$
\item $11^{3} \cdot 25^{3} \div 125^{3}$
\item $(-4)^{9} \div (-2)^{9}\div 81^{3}$\\\\
\textbf{Observaci�n:} $-a^{n} \not = (-a)^{n}$ \\
Ejemplos: \\
$(-5)^{4}=(-5) \cdot (-5) \cdot (-5)\cdot (-5) = 625$ \\
$-5^{4}=-(5 \cdot 5 \cdot 5 \cdot 5) =-(625)=-625$
\item $ \left( \dfrac{2}{3} \right)^{3}$
\item $\left( \left( \dfrac{3}{4} \right)^{3} \cdot \left( \dfrac{6}{7}\right)^{3}   \right)$
\item $\left(  \left( \dfrac{-1}{3}\right)^{-2} \cdot \left( \dfrac{4}{-6} \right)^{2} \div \left( \dfrac{9}{2}\right)^{-(-2)} \right)$
\end{enumerate}
\end{document}