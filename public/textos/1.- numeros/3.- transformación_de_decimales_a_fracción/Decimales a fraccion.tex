\documentclass[12pt,letterpaper]{article}
\usepackage[latin1]{inputenc}
\usepackage[spanish]{babel}
\usepackage{amsmath}
\usepackage{amsfonts}
\usepackage{amssymb}
\usepackage{makeidx}
\usepackage{graphicx}
\usepackage[left=2cm,right=2cm,top=2cm,bottom=2cm]{geometry}
\author{Nicol�s Gonz�lez Mart�nez}
\title{Transformaci�n de Decimales (Finitos e Infinitos a Fracci�n)}
\begin{document}
\maketitle
\begin{enumerate}
\item Decimal Finito a Fracci�n: \\
Para realizar esta conversi�n se toma el numero completo y se coloca en el numerador de la fracci�n, luego se coloca en el denominados un 1 y tantos ceros como decimales tenga el n�mero. \\\\
$3,715= \dfrac{3715}{1000}=\dfrac{743}{200}$
\item Decimal Infinito Semi-Peri�dico a Fracci�n: \\
Para realizar esta conversi�n se toma el numero completo y se coloca en el numerador de la fracci�n, luego se le resta lo que no pertenece al periodo del decimal, y para finalizar se colocan en el denominador tantos nueves como cifras tengo el periodo y tantos ceros como decimales no periodicos existan.  \\\\
$1,7 \overline{25}= \dfrac{1725-17}{990}=\dfrac{1708}{990}$
\item Decimal Infinito Semi-Peri�dico a Fracci�n: \\
Para realizar esta conversi�n se toma el numero completo y se coloca en el numerador de la fracci�n, luego se le resta lo no periodico para finalizar con colocar en el denominador tantos nueves como cifras existan en el periodo.  \\\\
$1,\overline{41}= \dfrac{141-1}{99}=\dfrac{140}{99}$
\end{enumerate}  
\section*{Gu�a a Trabajar}
Transforme a fracci�n los siguientes decimales 
\begin{enumerate}
\item $12,4$
\item $8,41$
\item $-45,8$
\item $-9,001$
\item $12,02\overline{32}$
\item $0,0\overline{01}$
\item $12,31\overline{4}$
\item $11,\overline{11}$
\item $-4,13\overline{49}$
\item $4,3\overline{142}$
\end{enumerate}
\end{document}