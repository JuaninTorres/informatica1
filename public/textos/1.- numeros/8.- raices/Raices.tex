\documentclass[12pt,letterpaper]{article}
\usepackage[latin1]{inputenc}
\usepackage[spanish]{babel}
\usepackage{amsmath}
\usepackage{amsfonts}
\usepackage{amssymb}
\usepackage{makeidx}
\usepackage{graphicx}
\usepackage[left=2cm,right=2cm,top=2cm,bottom=2cm]{geometry}
\author{Nicol�s Gonz�lez Mart�nez}
\title{Ra�ces}
\begin{document}
\maketitle
\textbf{Definici�n:} Si $n$ es un entero par positivo y $a$ es un rel positivo, entonces $\sqrt[n]{a}$ es un �nico real $b$, positivo, tal que $b^{n}=a$. \\\\
\textbf{Definici�n:} Si $n$ es n entero impar positivo y $a$ es un real cualquiera, entonces $\sqrt[n]{a}$ es un �nico real $b$, tal que, $b^{n}=a$. \\\\
Observaciones: 
\begin{enumerate}
\item si $n$ es un entero par positivo y $a$ es un real negativo, entonces $\sqrt[n]{a}$ \textbf{NO ES REAL}
\item la expresi�n $\sqrt[n]{a^{k}}$ con $a$ real positivo, puede expresarse como una potencia de exponente fraccionario, es decir $\sqrt[n]{a^{k}}=a^{\frac{k}{n}}$
\item $\sqrt{a^{2}}=\vert a \vert$, para todo n�mero real
\end{enumerate}
\textbf{Propiedades de las Ra�ces}
\begin{enumerate}
\item $\sqrt[n]{a} \cdot \sqrt[n]{b} = \sqrt[n]{a \cdot b}$
\item $\dfrac{\sqrt[n]{a}}{\sqrt[n]{b}} = \sqrt[n]{\dfrac{a}{b}}$
\item $\sqrt[n]{a^{m}}=(\sqrt[n]{a})^{m}$
\item $\sqrt[n]{\sqrt[m]{a}}=\sqrt[nm]{a}$
\item $\sqrt[n]{a} \cdot \sqrt[m]{b} = \sqrt[nm]{a^{m} \cdot b^{n}}$
\item $\sqrt[n]{b^{n}\cdot a}= b\cdot \sqrt[n]{a}$
\end{enumerate}
\textbf{Calcule el valor de:}
\begin{enumerate}
\item $\sqrt[4]{16}+\sqrt[5]{32}-\sqrt[3]{-27}$
\item $\sqrt{18}+\sqrt{50}-\sqrt{8}-2\sqrt{2}$
\item $\sqrt{0,0036}-\sqrt{0,01}$
\item Si $m=-3$ y $n=-2$, calcule $\sqrt{(m-n)^{2}}$
\item Si $a\bigtriangleup b= \sqrt[3]{b^{a}}$ entonces $\dfrac{3}{2} \bigtriangleup 25$ es
\item $\sqrt{\dfrac{32}{3}} \cdot \sqrt{\dfrac{27}{2}}$
\item $\sqrt{4\dfrac{3}{8}}\cdot \sqrt{\dfrac{7}{10}}$
\item Si $\sqrt{5} \approx 2$, entonces el valor de $\dfrac{\sqrt{108}+\sqrt{60}}{\sqrt{12}}$ es aproximadamente
\item Si $a+b=-\sqrt{3}$ y $ab=1-\sqrt{3}$, entonces $a^{2}+b^{2}$ es
\item $ \sqrt{(-3^{3})^{2}}$ equivale a
\item Demuestre que $5\sqrt{\dfrac{1}{5}\sqrt[3]{125}}$ es igual a 5
\item Demuestre que $\sqrt{\sqrt[3]{\sqrt[4]{8^{8}}}}$ es igual a 2
\item $\sqrt{27} \cdot \sqrt[3]{16}$
\item Si $m=\sqrt[8]{ab^{3}}$ y $n=\sqrt[6]{3a^{2}b^{3}}$, �Cual de las siguientes afirmaciones es siempre verdadera?
\begin{enumerate}
\item $\dfrac{n}{m}=\sqrt[24]{3^{4}a^{5}b^{3}}$
\item $\dfrac{m}{n}=\sqrt[48]{\dfrac{1}{3a}}$
\item $m \cdot n= \sqrt[48]{3a^{3}b^{6}}$
\item $m+n= \sqrt[24]{3^{4}a^{11}b^{21}}$
\end{enumerate} 
\end{enumerate}
\end{document}