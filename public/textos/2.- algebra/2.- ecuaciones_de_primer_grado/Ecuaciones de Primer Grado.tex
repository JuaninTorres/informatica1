\documentclass[12pt,letterpaper]{article}
\usepackage[latin1]{inputenc}
\usepackage[spanish]{babel}
\usepackage{amsmath}
\usepackage{amsfonts}
\usepackage{amssymb}
\usepackage{makeidx}
\usepackage{graphicx}
\usepackage[left=2cm,right=2cm,top=2cm,bottom=2cm]{geometry}
\author{Nicol�s Gonz�lez Mart�nez}
\title{Ecuaciones de Primer Grado}
\begin{document}
\maketitle
Una ecuaci�n se define como la igualdad $ax+b=0$ en donde $a \in \mathbb{R}^{*}$ y $b \in \mathbb{R}$. Lo que queremos encontrar es el valor de $x$ para que la igualdad se cumpla, en este caso la soluci�n de esta ecuaci�n seria $x=\dfrac{-b}{a}$ \\\\
\textbf{Ejemplos:} 
\begin{enumerate}
\item resolver la ecuaci�n $7x+3=10$ \\\\
soluci�n: $7x+3=10 \quad / (-3) \\
\Longleftrightarrow 7x=7 \quad / (\div 7) \\
\Longleftrightarrow x=1$
\item resolver la ecuaci�n $ \dfrac{x^{2}-6}{2}-\dfrac{2x^{2}+4x}{4}=5$ \\\\
soluci�n: $ \dfrac{x^{2}-6}{2}-\dfrac{2x^{2}+4x}{4}=5$ \\
$\Longleftrightarrow \dfrac{4(x^{2}-6)-2(2x^{2}+4x)}{8}=5$ \\
$\Longleftrightarrow \dfrac{4x^{2}-24-4x^{2}-8x}{8}=5$ \\
$\Longleftrightarrow \dfrac{-24-8x}{8}=5 \quad /(\cdot 8)$ \\
$\Longleftrightarrow \dfrac{8\cdot(-24-8x)}{8}=5\cdot 8$ \\
$\Longleftrightarrow -24-8x=40 \quad (+24)$ \\
$\Longleftrightarrow -8x=40+24$ \\
$\Longleftrightarrow -8x=64 \quad (\div 8)$ \\
$\Longleftrightarrow -x=8 \quad (\cdot -1)$ \\
$\Longleftrightarrow x=-8$
\end{enumerate}
\textbf{Resolver las siguientes ecuaciones}
\begin{enumerate}
\item $x(2x-3)-3(5-x) = 83$
\item $(2x+5)(2x-5) = 11$
\item $2(7 + x) + 3(7 ? x) = 130$
\item $(2x -3)(3x -4) - (x- 13)(x - 4) = 40$
\item $(3x - 4)(4x - 3) - (2x - 7)(3x - 2) = 214$
\item $8(2-x)^{2}=2(8-x)^{2}$
\item $\dfrac{x^{2}-6}{2}-\dfrac{2x^{2}+4}{4}=5$
\item $2x+10=16$
\item $5x+8=7x-32$
\item $5(x+2)=1+\dfrac{x}{2}$

\end{enumerate}
\end{document}